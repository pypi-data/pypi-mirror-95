\documentclass[
	superscriptaddress,
	twocolumn,
	aps, prl
]{revtex4-1}

\usepackage{amsmath}
\usepackage{amssymb}
\usepackage{hyperref}
\usepackage{graphicx}

% This file defines common macros used in many LaTeX files.
% They are collected here for convenience use.

% packages
\usepackage{amsmath}
\usepackage{xcolor}
\usepackage{dsfont} % double stroke font
\usepackage{xspace}
\usepackage{xparse}


% abbrevations - use \mbox instead of \nolinebreak since this works in captions, too
\newcommand{\ie}{\mbox{i.\hspace{0.125em}e.}\@\xspace}
\newcommand{\eg}{\mbox{e.\hspace{0.125em}g.}\@\xspace}
\newcommand{\etal}{\mbox{et al.}\@\xspace}
\newcommand{\kb}{k_\textnormal{B}}
\newcommand{\kB}{k_\textnormal{B}}
\newcommand{\kbT}{k_\textnormal{B}T}
\newcommand{\kBT}{k_\textnormal{B}T}
\newcommand{\const}{\text{const.}}

% colors, which are often used
\definecolor{plot1}{RGB}{6,115,183}
\definecolor{plot2}{RGB}{255,118,0}
\definecolor{plot3}{RGB}{0,169,25}
\definecolor{plot4}{RGB}{230,0,28}
\definecolor{plot5}{RGB}{0,0,0}
\definecolor{plotBlue}{RGB}{6,115,183}
\definecolor{plotOrange}{RGB}{255,118,0}
\definecolor{plotGreen}{RGB}{0,169,25}
\definecolor{plotRed}{RGB}{230,0,28}

% draft commands
\newcommand{\fixme}[1]{\textcolor{red}{FIXME: #1}}
\newcommand{\todo}[1]{\textcolor{red}{TODO: #1}}
\newcommand{\edit}[1]{\textcolor{blue}{#1}}

% references - use \mbox instead of \nolinebreak since this works in captions, too
\newcommand{\Eqref}[1]{\mbox{Eq.\hspace{0.25em}\eqref{#1}}}
\newcommand{\Eqsref}[1]{\mbox{Eqs.\hspace{0.25em}\eqref{#1}}}
\newcommand{\figref}[1]{\mbox{Fig.\hspace{0.25em}\ref{#1}}}
\newcommand{\figsref}[1]{\mbox{Figs.\hspace{0.25em}\ref{#1}}}
\newcommand{\tabref}[1]{\mbox{Tab.\hspace{0.25em}\ref{#1}}}
\newcommand{\refcite}[1]{\mbox{Ref.\hspace{0.25em}\cite{#1}}}

% math constructs
\newcommand{\im}{{\mathbf{i}}}
\newcommand{\diff}{\text{d}}
\newcommand{\pfrac}[2]{\frac{\partial #1}{\partial #2}}
\newcommand{\deltafrac}[2]{\frac{\delta #1}{\delta #2}}
\newcommand{\difffrac}[2]{\frac{\diff #1}{\diff #2}}
\newcommand{\Difffrac}[2]{\frac{\text{D} #1}{\text{D} #2}}
\newcommand{\takenat}[2]{\left.{#1}\right|_{#2}}
\newcommand{\abs}[1]{|{#1}|}
\newcommand{\Abs}[1]{\left| {#1} \right|}
\newcommand{\norm}[1]{\|{#1}\|}
\newcommand{\Norm}[1]{\left\| {#1} \right\|}
\newcommand{\order}[1]{\mathcal{O}(#1)}
\newcommand{\Order}[1]{\mathcal{O}\left(#1\right)}
\newcommand{\set}[1]{\{{#1}\}}
\newcommand{\Set}[1]{\left\{{#1}\right\}}
\newcommand{\unity}{\mathds{1}}
\newcommand{\identity}{\mathds{1}}
\newcommand{\transpose}[1]{{#1}^\top}
\newcommand{\scalarproduct}[2]{\langle #1 ,\, #2 \rangle}
\newcommand{\Scalarproduct}[2]{\left\langle #1 ,\, #2 \right\rangle}

% math font sizes
\newcommand{\msmall}[1]{\mbox{\small $#1$}}
\newcommand{\mscriptsize}[1]{\mbox{\scriptsize $#1$}}
\newcommand{\mtiny}[1]{\mbox{\tiny $#1$}}

%\newcommand{smallequation}[1]{\begingroup\small#1\endgroup}
\newenvironment{smallequation}{\begingroup\small\ignorespaces}{\endgroup\ignorespacesafterend}

% math sets
\newcommand{\natnum}{\mathds{N}}
\newcommand{\integers}{\mathds{Z}}
\newcommand{\rationals}{\mathds{Q}}
\newcommand{\reals}{\mathds{R}}
\newcommand{\complexes}{\mathds{C}}

% math operators
\DeclareMathOperator{\arctanh}{arctanh}
\DeclareMathOperator{\sech}{sech}
\DeclareMathOperator{\sgn}{sgn}
%\DeclareMathOperator{\max}{max}
%\DeclareMathOperator{\min}{min}
\DeclareMathOperator{\std}{std}
\DeclareMathOperator{\var}{var}
\DeclareMathOperator{\cov}{cov}
\DeclareMathOperator{\CV}{Cv}
\DeclareMathOperator{\STD}{STD}
\DeclareMathOperator{\SEM}{SEM}
\DeclareMathOperator{\erf}{erf}
\DeclareMathOperator{\erfc}{erfc}
\newcommand{\mean}[1]{\langle #1 \rangle}
\newcommand{\Mean}[1]{\left\langle #1 \right\rangle}
\renewcommand\Re{\operatorname{\mathfrak Re}}
\renewcommand\Im{\operatorname{\mathfrak Im}}
\newcommand{\Floor}[1]{\left\lfloor #1 \right\rfloor}
\newcommand{\floor}[1]{\lfloor #1 \rfloor}
\newcommand{\Ceil}[1]{\left\lceil #1 \right\rceil}
\newcommand{\ceil}[1]{\lceil #1 \rceil}
\newcommand{\Nabla}{\vect\nabla}

% special words
\newcommand{\gene}[1]{\mbox{\textit{#1}}}

% special symbols
\DeclareFontFamily{U}{mathx}{\hyphenchar\font45}
\DeclareFontShape{U}{mathx}{m}{n}{<-> mathx10}{}
\DeclareSymbolFont{mathx}{U}{mathx}{m}{n}
\DeclareMathAccent{\widebar}{0}{mathx}{"73}
\newcommand{\vect}{\boldsymbol}
\newcommand{\mat}[1]{\underline{\underline{#1}}}

% symbols for perturbation analysis
\newcommand{\eps}{\varepsilon}
\newcommand{\pert}{\widehat}
\newcommand{\steady}{\widebar}
\newcommand{\pprime}{\prime\prime}
\newcommand{\ppprime}{\prime\prime\prime}

% roman numbers with bars
\def\barroman#1{\sbox0{#1}\dimen0=\dimexpr\wd0+1pt\relax
  \makebox[\dimen0]{\rlap{\vrule width\dimen0 height 0.06ex depth 0.06ex}%
    \rlap{\vrule width\dimen0 height\dimexpr\ht0+0.03ex\relax 
            depth\dimexpr-\ht0+0.09ex\relax}%
    \kern.5pt#1\kern.5pt}}
 
% new environments
\ExplSyntaxOn
\NewDocumentEnvironment{salign} {o}
{\subequations
      \IfNoValueTF{#1} {}{\label{#1}}
      \align}
{\endalign\endsubequations}
\ExplSyntaxOff

% hyphenation

\hyphenation{non-equilibrium}



\newcommand{\dx}{\Delta x}
\newcommand{\dr}{\Delta r}
\renewcommand{\L}{_\mathrm{L}}
\newcommand{\R}{_\mathrm{R}}
\newcommand{\dom}{\Omega}
\newcommand{\bndry}{\partial\Omega}

\begin{document}

\title{Boundary conditions in the finite difference python package `py-pde`}
\author{David Zwicker}
\date{\today}

\begin{abstract}
We here document the differential operators and the associated boundary conditions that are used in the finite difference approximation.
We also derive special versions of the Laplace operator that conserves the total mass.
\end{abstract}


\maketitle
\tableofcontents


\section{General considerations}
We start with general considerations that are independent of the grid type.

\subsection{Differential operators}
We first define the differential operators that act on fields defined in a domain~$\dom$.
In particular, we discuss the boundary conditions that can be enforced with each type of operator.

\paragraph{Laplace operator:}
The Laplace operator $\partial_\alpha^2 c$ is defined to act on scalar fields~$c(\vect r)$.
Possible boundary conditions are
\begin{salign}[eqn:bc_laplace]
	\text{value:} &&	c &= A
\\
	\text{derivative:} && n_\alpha \partial_\alpha c &= B
\end{salign}
which are applied at positions $\vect r \in \bndry$.
Consequently, either the value~$A$ of the field can be determined, \ie, $c(\vect r) = A$ at $\vect r \in \bndry$, or the derivative can be specified, $n_\alpha \partial_\alpha c = B$, where $n_\alpha$ is the outwards oriented normal vector at the boundary.


\paragraph{Gradient operator:}
The gradient operator $\partial_\alpha c$ acts on a scalar field $c(\vect r)$ yielding a vector field.
Possible boundary conditions are
\begin{salign}[eqn:bc_gradient]
	\text{value:} &&	c &= A
\\
	\text{derivative:} && n_\alpha \partial_\alpha c &= B
\end{salign}

\paragraph{Divergence operator:}
The divergence operator $\partial_\alpha v_\alpha$ acts on a vector field $v_\alpha(\vect r)$ with possible boundary conditions
\begin{salign}[eqn:bc_divergence]
	\text{value:} && n_\alpha v_\alpha  &= A
\\
	\text{derivative:} && \partial_\alpha v_\alpha &= B
\end{salign}
Note that the Dirichlet condition in this case imposes the dot product between the normal vector and the vector field, which in a simple 1d system implies $v_1(x=x\L)=-A$ and $v_1(x=x\R)=A$ for the lower and upper boundary, respectively.

\paragraph{Vector gradient operator:}
The vector gradient  operator $\partial_\alpha v_\beta$ acts on a vector field $v_\alpha(\vect r)$ and yields a tensorial field.
This operator can be though of as a gradient applied to each component of $v_\beta$ and the boundary conditions thus are also applied to each component,
\begin{salign}
	\text{value:} &&	v_\alpha &= A_\alpha
\\
	\text{derivative:} && n_\alpha \partial_\alpha v_\beta &= B_\beta
\end{salign}

\paragraph{Tensor divergence operator:}
The vector gradient  operator $\partial_\beta t_{\alpha\beta}$ acts on a tensor field $t_{\alpha\beta}(\vect r)$ and yields a vector field.
This operator can be though of as a divergence applied along the $\beta$-direction to each $\alpha$-component.
The boundary conditions thus are applied to each component,
\begin{subequations}
\begin{align}
	\text{value:} &&	n_\beta t_{\alpha\beta} &= A_\alpha
\\
	\text{derivative:} &&\partial_\beta t_{\alpha\beta}&= B_\alpha
\end{align}
\end{subequations}

\paragraph{Operators not implemented:}
Currently not implemented are the following operators
\begin{subequations}
\begin{align}
	\text{Curl:} &&
		\epsilon_{\alpha\beta\gamma} \partial_\beta v_\gamma
\\
	\text{Vector laplacian:} &&
		\partial_\alpha^2 v_\beta
\\
	\text{Material derivative:}&&
		w_\alpha \partial_\alpha v_\beta
\end{align} 
\end{subequations}


%\subsection{Boundary conditions of compound operators}
%The Laplace operator can be written as a gradient operator followed by a divergence operator, $\partial_\alpha^2 c = \partial_\alpha(\partial_\alpha c)$.
%When implementing the involved operators in finite difference, one necessarily needs to determine boundary conditions for all three differential operators.
%Clearly, boundary conditions given in \Eqref{eqn:bc_laplace} for the Laplace operator can also be applied to the gradient operator, which employs the same conditions; see \Eqref{eqn:bc_gradient}.
%Conversely, the boundary conditions on the divergence operator specify 



\section{Cartesian Grid}
We consider a linear domain of length $L$ discretized by $N$ support points.
We place these points equidistantly at $x_n = (n+\frac12)\dx$ for $n=0,1,\ldots, N-1$ using the discretization $\dx=L/N$.
Any function $y=f(x)$ is then represented by discretized values $y_n = f(x_n)$.

In the finite difference schemes we discuss, we require differential operators of first and second order.
Targeting second order accuracy, we obtain these as 
\begin{subequations}
\begin{align}
	f'(x_n) &\sim \frac{y_{n+1} - y_{n-1}}{2\dx}
\\	
	f''(x_n) &\sim \frac{y_{n+1} - 2 y_n + y_{n-1}}{\dx^2}
\end{align}
\end{subequations}
where $\sim$ denotes the finite difference approximation.

These derivatives can only be evaluated directly for $n=1,2,\ldots, n-2$, while the boundary points $n=0$ and $n=N-1$ require knowledge of the value at the respective virtual points $x_{-1}$ and $x_N$ outside the domain.
We can derive expressions for the associated function values $y_{-1}$ and $y_N$  taking the boundary conditions in to account, which typically specify the value at the boundary or the (outward) derivative.
A particular simple case are periodic boundary conditions, where we have $y_{-1} = y_{N-1}$ and $y_N = y_0$, allowing the above formula to be used everywhere.
For more complicated boundary conditions, we need to treat both sides separately.

\subsection{Lower boundary}
Let us first consider conditions at the lower boundary, which we place at $x=0$.
We thus need to evaluate the virtual support point $y_{-1}$ from the boundary condition.
Using estimates for the value and the derivative at the boundary,
\begin{align}
	f(0) &\sim \frac{y_0 + y_{-1}}{2}
&
	f'(0) &\sim \frac{y_0 - y_{-1}}{\dx}
\end{align}
we can now handle several boundary conditions.

\paragraph{Dirichlet (Constant value):}
Assuming the boundary condition $f(0) = a$, we obtain $y_{-1} = 2a - y_0$.

\paragraph{Neumann (Constant derivative):}
Here, we consider the boundary condition $-f'(0) = b$, which specifies the outward derivative to equal the value $b$.
We obtain $y_{-1} = y_0 + b \dx$.

\paragraph{Robin (Mixed):}
Assuming the boundary condition $-f'(0) + c f(0) = b $, we obtain $y_{-1} = A - B y_0$
where 
\begin{align}
	A &= \frac{2\dx}{c \dx + 2} b
&
	B &= \frac{c\dx - 2}{c \dx + 2}
\end{align}
%Hence,
%\begin{align}
%	\takenat{\pfrac{^2 f}{x^2}}{x_0} \sim \frac{A - (2 + B) y_0 + y_{1}}{\dx^2}
%\end{align}
As a sanity check, we obtain Dirichlet conditions in the limit $b \rightarrow \infty$ with $b/c = a$, where $A=2a, B=1$.
Likewise, we recover Neumann condition for $c=0$.

\paragraph{Summary:}
Taken together, we can express the value of the support point as
\begin{align}
	y_{-1} &= \alpha\L + \beta\L y_{k\L}
	\;,
\end{align}
where the coefficients $\alpha\L$, $\beta\L$, and $k\L$ are taken from \tabref{tab:cartesian_lower}.


\begin{table}[t]
\caption{\label{tab:cartesian_lower}%
Coefficients for lower cartesian boundary
}
\begin{ruledtabular}
	\begin{tabular}{cccc}
		Boundary condition & Offset $\alpha\L$ & Pre-factor $\beta\L$  & Index $k\L$\\
		\colrule
		Periodic & $0$ & $1$ & $N-1$ \\
		Dirichlet & $2a$ & $-1$ & $0$ \\
		Neumann & $b\dx$ & $1$  & $0$ \\
		Robin & $\frac{2b\dx}{2 + c \dx}$ &
				$\frac{2 - c\dx}{2 + c \dx}$ & $0$ \\
	\end{tabular}
\end{ruledtabular}
\end{table}


\subsection{Upper boundary}
The upper boundary at $x=L$ can be treated analogously to the lower one by introducing a virtual point at $x=x_N$.
Consequently, we find the following conditions for the virtual support point $y_N$:
\paragraph{Dirichlet (Constant value):}
Assuming the boundary condition $f(L) = a$, we obtain $y_N = 2a - y_{N-1}$.

\paragraph{Neumann (Constant derivative):}
Here, we consider the boundary condition $f'(L) = b$, which specifies the outward derivative to equal the value $b$.
We obtain $y_N = y_{N-1} + b \dx$.


Taken together, we express the value at the virtual support point as
\begin{align}
	y_N &= \alpha\R + \beta\R y_{k\R}
	\;,
\end{align}
where the coefficients $\alpha\R$, $\beta\R$, and $k\R$ are taken from \tabref{tab:cartesian_upper}.


\begin{table}[t]
\caption{\label{tab:cartesian_upper}%
Coefficients for upper cartesian boundary
}
\begin{ruledtabular}
	\begin{tabular}{cccc}
		Boundary condition & Offset $\alpha\R$ & Pre-factor $\beta\R$  & Index $k\R$\\
		\colrule
		Periodic & $0$ & $1$ & $0$ \\
		Dirichlet & $2a$ & $-1$ & $N-1$ \\
		Neumann & $b\dx$ & $1$  & $N-1$ \\
		Robin & $\frac{2b\dx}{2 + c \dx}$ &
				$\frac{2 - c\dx}{2 + c \dx}$ & $N-1$ \\
	\end{tabular}
\end{ruledtabular}
\end{table}


\section{Cylindrical Grid}
We here consider fields in $3$ dimensions that possess angular symmetry and thus only depend on the radial coordinate~$r$ and the axial coordinate~$z$.
Clearly, the axial coordinate behaves exactly as the Cartesian ones described above, so we here only have to deal with the polar symmetry.
Similar to Cartesian grids, we discretize the associated radial coordinate as $r_n = r_0 + (n + \frac12) \dr$, where $r_0$ is the inner radius, which often will be zero.

The radial part of the gradient operator in cylindrical coordinates is simply given by $\partial_r f(r)$ and thus obeys the same discretization as in a Cartesian grid.
The divergence of a vector field $v_\alpha(r) = \rho(r) \vect e_r$ reads
$\partial_\alpha v_\alpha(r) =  \rho'(r) + \frac{\rho(r)}{r}$, implying the discretized version
\begin{align}
	\partial_\alpha v_\alpha \sim
		\frac{y_{n+1} - y_{n-1}}{2\dr}
		+ \frac{y_n}{r_n}
\end{align}
Moreover, the Laplace operator in spherical coordinates reads $\nabla^2 f(r) = f''(r) + \frac{f'(r)}{r}$, which in the discretized version becomes
\begin{align}
	\partial_\alpha^2 f \sim
		\frac{y_{n+1} - 2 y_n + y_{n-1}}{\dr^2}
		+ \frac{y_{n+1} - y_{n-1}}{2 r_n \dr}
\end{align}

\subsection{Boundary condition at the origin}
Due to the symmetry, only vanishing derivatives are allowed at the inner boundary at the origin, implying the virtual support point $y_{-1} = y_0$.
Hence, the differential operators become
\begin{subequations}
\begin{align}
	\takenat{\partial_\alpha v_\alpha(r)}{r=0} &\sim
		\frac{y_1 - y_0}{2\dr}
		+ \frac{2y_0}{\dr}
\\
	\takenat{\partial_\alpha^2 f(r)}{r=0} &\sim
		2\frac{y_1 - y_0}{\dr^2}
%		\frac{y_1 - 2 y_0 + y_0}{\dr^2}
%		+ \frac{2(y_1 - y_0)}{\dr^2}
%		\frac{3y_1 -  3y_0}{\dr^2}
\end{align}
\end{subequations}


\subsection{Boundary condition at the outer side}
At the outer boundary, we can impose boundary conditions similar to the Cartesian grid.

\paragraph{Dirichlet (Constant value):}
Assuming the boundary condition $f(R) = a$, we obtain $y_N = 2a - y_{N-1}$.

\paragraph{Neumann (Constant derivative):}
Here, we consider the boundary condition $f'(L) = b$, which specifies the outward derivative to equal the value $b$.
We obtain $y_n = y_{n-1} + b \dx$.



\section{Spherical Grid}
We here consider fields that a spherically symmetric in $3$ dimensions and thus only depend on the radial coordinate~$r$.
Similar to Cartesian grids, we discretize the radial coordinate as $r_n = (n + \frac12) \dr$.

The radial part fo the gradient operator in spherical coordinates is simply given by $\partial_r f(r)$ and thus obeys the same discretization as in a Cartesian grid.
The divergence of a vector field $v_\alpha(r) = \rho(r) \vect e_r$ reads
$\partial_\alpha v_\alpha(r) =  \rho'(r) + \frac{2\rho(r)}{r}$, implying the discretized version
\begin{align}
	\partial_\alpha v_\alpha \sim
		\frac{y_{n+1} - y_{n-1}}{2\dr}
		+ \frac{2y_n}{r_n}
\end{align}
Moreover, the naive implementation of Laplace operator in spherical coordinates reads $\nabla^2 f(r) = f''(r) + \frac{2f'(r)}{r}$, which in the discretized version becomes
\begin{align}
	\partial_\alpha^2 f \sim
		\frac{y_{n+1} - 2 y_n + y_{n-1}}{\dr^2}
		+ \frac{y_{n+1} - y_{n-1}}{r_n \dr}
\end{align}
Note that this form of the Laplace operator is not conservative, \ie, the discretized version of the integral $\int \partial_\alpha^2 f \diff r$  does not vanish.


\subsection{Boundary condition at the origin}
Due to the symmetry, only vanishing derivatives are allowed at the inner boundary at the origin, implying the virtual support point $y_{-1} = y_0$.
Hence, the differential operators become
\begin{subequations}
\begin{align}
	\takenat{\partial_\alpha v_\alpha(r)}{r=0} &\sim
		\frac{y_1 - y_0}{2\dr}
		+ \frac{4y_0}{\dr}
\\
	\takenat{\partial_\alpha^2 f(r)}{r=0} &\sim
%		\frac{y_1 - 2 y_0 + y_0}{\dr^2}
%		+ \frac{2(y_1 - y_0)}{\dr^2}
		3\frac{y_1 -  y_0}{\dr^2}
\end{align}
\end{subequations}
Note that the latter expression applies to both the non-conservative and the conservative form of the Laplacian.

\subsection{Boundary condition at the outer side}
At the outer boundary, we can impose boundary conditions similar to the Cartesian grid.

\paragraph{Dirichlet (Constant value):}
Assuming the boundary condition $f(R) = a$, we obtain $y_N = 2a - y_{N-1}$.

\paragraph{Neumann (Constant derivative):}
Here, we consider the boundary condition $f'(L) = b$, which specifies the outward derivative to equal the value $b$.
We obtain $y_n = y_{n-1} + b \dx$.

\subsection{Conservative operator}

We call a discrete Laplace operator conservative when it preserves the desirable conservation equation
\begin{align}
	\int_\Omega \partial_\alpha^2 f \diff V
		= \oint_{\partial\Omega} \partial_\alpha f n_\alpha \diff S= 0
\end{align}
when Neumann boundary conditions are imposed on the function $f$.
A conservative operator can be derived by integrating the definition of the Laplace operator in spherical cooridnates over one spherical shell from $r=r_n - \dr/2$ to $r=r_n  + \dr/2$,
\begin{align}
%	r^2 \partial_\alpha^2 f &= \partial_r\bigl(r^2 \partial_r f(r)\bigr)
%\\
	\int_{r_{n - \frac12}}^{r_{n + \frac12}} r^2 \partial_\alpha^2 f \diff r
		&= \int_{r_{n - \frac12}}^{r_{n + \frac12}} \partial_r\bigl(r^2 \partial_r f(r)\bigr) \diff r
	\;,
\end{align}
where we skipped the factor $4\pi$ stemming from the angle integration.
Assuming that the quantity $\partial_\alpha^2 f$ is constant across the discretization cell, we thus find
\begin{align}
	V_n \partial_\alpha^2 f %\left[ \frac{r^3}{3} \right]_{r_n - \dr/2}^{r_n + \dr/2}
		&\sim \Bigl[r^2 \partial_r f(r)\Bigr]_{r_n - \dr/2}^{r_n + \dr/2}
\end{align}
where we introduced the (scaled) shell volumes
\begin{align}
%	V_n &= \frac{\dr^3}{3}\bigl[(n + 1)^3 - n^3\bigr]
	V_n &= \frac{r_{n + \frac12}^3 - r_{n - \frac12}^3}{3}
	= \dr \left(	r_n^2 + \frac{\dr^2}{12} \right)
\;.
\end{align}
Consequently,
\begin{align}
	\partial_\alpha^2 f
%		&\sim \dr^2 \frac{(n + 1)^2 f'\left(r_n + \frac{\dr}{2}\right) - n^2 f'\left(r_n - \frac{\dr}{2}\right)}{V_n}
		&\sim \frac{r_{n+\frac12}^2 f'\left(r_{n+\frac12}\right) - r_{n-\frac12}^2 f'\left(r_{n-\frac12}\right)}{V_n}
\end{align}
where the derivatives $f'(r)$ are evaluated at the midpoints and can thus be represented as
\begin{subequations}
\begin{align}
	f'\left(r_{n + \frac12}\right) \sim \frac{y_{n+1} - y_n}{\dr}
\\
	f'\left(r_{n - \frac12}\right) \sim \frac{y_n - y_{n-1}}{\dr}
\end{align}
\end{subequations}
In the special case where the spherical grid does not have a hole ($r_0=0$), we then arrive at
\begin{align}
	\partial_\alpha^2 f
		&\sim \frac{3}{\dr^2} \, \frac{(n + 1)^2 (y_{n+1} - y_n) - n^2 (y_n - y_{n-1})}{(n + 1)^3 - n^3}
	\;,
\end{align}
which is a conservative discretization of the spherical Laplacian by construction.

Similarly, we can consider a spherical grid where fields depend on the polar angle~$\theta$ while still keeping azimuthal symmetry, i.e., assuming no dependence on $\phi$.
The condition for a conservative Laplace operator then reads
\begin{widetext}
\begin{multline}
	\iint_{\text{cell}(n, m)} r^2 \sin\theta \, \partial_\alpha^2 f  \diff r \diff \theta
=
	\iint_{\text{cell}(n, m)} \biggl(
		\sin\theta  \partial_r\bigl[r^2 \partial_r f(r, \theta)\bigr]
		+ \partial_\theta \bigl[ \sin\theta \partial_\theta  f(r, \theta)\bigr]
	\biggr)\diff r \diff \theta
\\=
	\Bigl(\cos\theta_{m-\frac12} - \cos\theta_{m + \frac12}\Bigr)
		\int_{r_{n - \frac12}}^{r_{n + \frac12}}\left(
			 \partial_r\bigl[r^2 \partial_r f(r, \theta)\bigr]
		\right)  \diff r
%\notag\\&\quad
	+ \Delta r
		\int_{\theta_{m - \frac12}}^{\theta_{m + \frac12}}
			 \partial_\theta \bigl[ \sin\theta \partial_\theta  f(r, \theta)\bigr]	 \diff\theta
\\=
	\Bigl(\cos\theta_{m-\frac12} - \cos\theta_{m + \frac12}\Bigr)
			\bigl[r^2 \partial_r f(r, \theta)\bigr]_{r_{n - \frac12}}^{r_{n + \frac12}}
	+  \Delta r
		\bigl[ \sin\theta \partial_\theta  f(r, \theta)\bigr]_{\theta_{m - \frac12}}^{\theta_{m + \frac12}}
%\\
%	\partial_\alpha^2 f \left[ \frac{r^3}{3} \right]_{r_n - \dr/2}^{r_n + \dr/2}
%		&\sim \Bigl[r^2 \partial_r f(r)\Bigr]_{r_n - \dr/2}^{r_n + \dr/2}
\end{multline}
\end{widetext}
Here we assumed that we can neglect angular dependencies in the radial integral and vice versa.
Using the (scaled) volume of a grid cell,
\begin{align}
	V_{n,m} =
		\left(\cos\theta_{m-\frac12} - \cos\theta_{m + \frac12}\right)
		\frac{r_{n + \frac12}^3 - r_{n - \frac12}^3}{3}
%\notag\\=
%	\left(\cos\theta_{m-\frac12} - \cos\theta_{m + \frac12}\right)
	\;,
\end{align}
we can write this as
\begin{multline}
	V_{n,m} \partial_\alpha^2 f
=\Bigl(\cos\theta_{m-\frac12} - \cos\theta_{m + \frac12}\Bigr)
			\bigl[r^2 \partial_r f(r, \theta)\bigr]_{r_{n - \frac12}}^{r_{n + \frac12}}
\\
	+  \Delta r
		\bigl[ \sin\theta \partial_\theta  f(r, \theta)\bigr]_{\theta_{m - \frac12}}^{\theta_{m + \frac12}}
\end{multline}

%\bibliographystyle{apsrev4-1}
%\bibliography{bibdesk.bib}

\end{document}  